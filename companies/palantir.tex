\cveventflat
{Software Engineer at }
{\textbf{Palantir Technologies}, \textit{London, United Kingdom}}
{Sep. 2020 - now}
{London, United Kingdom}
\newline
\href{https://www.palantir.com/}{\textbf{www.palantir.com}}
\begin{itemize}
    \item {
        Introduced
        \href{https://www.palantir.com/docs/foundry/administration/container-governance/}{automatic vulnerability scanning and security controls}
        for user-uploaded Docker containers in Foundry, allowing users to recall them based on found exposures.
        This green-lighted
        \href{https://www.palantir.com/docs/foundry/transforms-python/container-overview/}{a workflow},
        allowing users to use Docker containers as a part of their data pipeline.
    }
    \textbf{[Java, Docker, Trivy]}
    \item {
        Implemented
        \href{https://www.palantir.com/docs/foundry/code-repositories/aip-features/#code-autocomplete}{a tool similar to GitHub Copilot},
        which provides users with LLM-generated code completions including support for Foundry-specific code,
        resulting in ~50\% reduction in lead time to production.
    }
    \item {
        Created
        \href{https://www.palantir.com/docs/foundry/transforms-common/local-preview/}{a Gradle plugin},
        allowing power users to quickly preview Foundry datasets produced by their Spark code in IDEs of their
        choice, which greatly increased their iteration speed as this feature was previously available only in-browser.
    }
    \textbf{[Java, Python, Gradle]}
    \item {
        Achieved 30x performance boost for
        \href{https://www.palantir.com/docs/foundry/data-lineage/overview/}{Data Lineage},
        a tool that facilitates a holistic view of how data flows through the Foundry,
        by decreasing P99 of a BE endpoint powering it from ~30s to less than 1s.
    }
    \item {
        Seamlessly migrated
        \href{https://www.palantir.com/docs/foundry/data-integration/application-reference/#builds}{Job Tracker},
        a tool allowing users to view
        \href{https://www.palantir.com/docs/foundry/data-integration/builds/}{builds}
        happening in Foundry, to a new ElasticSearch index which improved its security and lowered costs.
    }
    \textbf{[Java, ElasticSearch]}
    \item {
        Delivered core BE functionality for
        \href{https://www.palantir.com/docs/foundry/maintaining-pipelines/monitoring-views-intro/}{"Monitoring at scale"}
        which is a highly scalable and low latency tool for monitoring resources in Foundry.
        It replaced multiple pre-existing ad-hoc tools and allowed monitoring of entire data pipelines in a set-and-forget way.
    }
    \textbf{[Java]}
    \item {
        As a part of a hack-week competition, implemented BE for a MVP version of
        \href{https://www.palantir.com/docs/foundry/hyperauto/overview/}{Palantir HyperAuto},
        a software-defined data integration (SDDI) tool for data systems like Salesforce and SAP.
        This allowed users to create end-to-end data pipelines based on thousands of datasets
        in just a few clicks, a process that would otherwise take days of manual work.
    }
    \textbf{[Java, SAP]}
    \item {
        Contributed to
        \href{https://www.palantir.com/docs/foundry/code-repositories/aip-features/#ai-error-enhancer}{AI Error Enhancer},
        a tool that provides comprehensive error explanations and suggested code changes in multiple apps across Foundry.
        This helped to reduce the number of customer-filed support tickets.
    }
    \textbf{(pending patent)}
    \item {
        Wrote
        \href{https://www.palantir.com/docs/foundry/}{Palantir's public docs}
        for most of the features mentioned in this document.
    }
\end{itemize}
