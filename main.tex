\documentclass[10pt,a4paper]{altacv}
\geometry{left=1cm,right=1cm,marginparwidth=6.8cm,marginparsep=1.2cm,top=1cm,bottom=1cm}

\usepackage[utf8]{inputenc}
\usepackage[T1]{fontenc}
\usepackage[default]{lato}

\definecolor{blue(pigment)}{rgb}{0.2, 0.2, 0.6}

\usepackage[colorlinks = true,
    linkcolor = blue(pigment),
    urlcolor  = blue(pigment),
    citecolor = blue(pigment),
    anchorcolor = blue(pigment)]{hyperref}


\definecolor{VividPurple}{HTML}{000000}
\definecolor{SlateGrey}{HTML}{242424}
\definecolor{LightGrey}{HTML}{202020}

\colorlet{heading}{VividPurple}
\colorlet{accent}{VividPurple}
\colorlet{emphasis}{SlateGrey}
\colorlet{body}{LightGrey}

\renewcommand{\itemmarker}{{\small\textbullet}}
\renewcommand{\ratingmarker}{\faCircle}

\newcommand{\custominfo}[4] {
    \includegraphics[scale=#2]{#1}
    \hspace{#3}
    #4
    \hspace{5mm}}

\newcommand{\codeforces}[1] {
    \custominfo{codeforces.png}{0.05}{0.6mm}{#1}}

\newcommand{\hackerrank}[1] {
    \custominfo{hackerrank.png}{0.018}{0.2mm}{#1}}

\newcommand{\leetcode}[1] {
    \custominfo{leetcode.png}{0.028}{0mm}{#1}}

\begin{document}



    \name{Dauren Abdykaparov}
    \tagline{} % moto
    \personalinfo{
    % \printinfo{symbol}{detail}
    % \mailaddress{}
    % \homepage{}
    % \twitter{}
    % \orcid{}
        \large
        \href{mailto:daurenabd@gmail.com}{\email{daurenabd@gmail.com}}
        \href{tel:821073508907}{\phone{+8210-9720-9607}}
        \href{https://linkedin.com/in/daurenabd}
        {\linkedin{linkedin.com/in/daurenabd}}
        \newline
        \large
        \href{https://github.com/daurenabd}
        {\github{github.com/daurenabd}}
        \href{https://leetcode.com/daurenabd/}
        {\leetcode{leetcode.com/daurenabd}}
    }
    %\href{mailto:daurenabd@gmail.com}{\email{daurenabd@gmail.com}}
    %\href{tel:821073508907}{\phone{+8210-7350-8907}}
    %\newline
    %\href{https://github.com/daurenabd}{\github{github.com/daurenabd}}
    %\href{https://codeforces.com/profile/YoungCoder}{\codeforces{codeforces.com/profile/YoungCoder}}
    %\href{https://hackerrank.com/daurenabd}{\hackerrank{hackerrank.com/daurenabd}}}

    \begin{fullwidth}
        \makecvheader
    \end{fullwidth}

    \large
    \cvsection{Experience}

    \cveventflat
    {Software Engineer at }
    {\textbf{Palantir}, \textit{London, United Kingdom}}
    {Sep. 2020}
    {London, United Kingdom}
    \newline
    \href{https://www.palantir.com/}{\textbf{www.palantir.com}}

    \medskip

    \cveventflat
    {Software Engineer at }
    {\textbf{Skelter Labs}, \textit{Seoul, South Korea}}
    {Mar. 2019 -- Aug. 2020}
    {Seoul, South Korea}
    \newline
    \href{https://www.skelterlabs.com/}{\textbf{www.skelterlabs.com}}
    \begin{itemize}
        \item Dived into Deep Learning research and engineering project for the first time, and had major contribution in creating a top-notch Korean Machine Reading Comprehension (MRC) engine.
        \item Achieved top ranking in \textit{Korean Question Answering Dataset (KorQuAD)} version 1 and 2
        \href{https://korquad.github.io/}{\textbf{[www.korquad.github.io]}}
        \item Introduced SpanBERT to the team. Implemented and optimized it to become the 1st rank in \textit{KorQuAD-v1} with F1 score of 95.15 as of Jan., 2020. Skelter Labs was the 1st rank for 6 months
        \textbf{[Python, Tensorflow, GCP]}
        \item Fine-tuned and optimized the engine to \textit{KorQuAD-v2}, which includes HTML tags with lists and tables, with significantly longer answer spans, achieving 88.09 F1 score to be ranked 1st on June, 2020
        \textbf{[Python, Tensorflow, GCP]}
        \item Created pretraining data generation pipeline for BERT-like models, enabling more frequent experiments with those models
        \textbf{[Python, Spark, Kubernetes]}
    \end{itemize}

    \medskip

    \cveventflat
    {Software Engineer Intern at }
    {\textbf{HENNGE}, \textit{Tokyo, Japan}}
    {Jan. 2019 -- Mar. 2019}
    {Tokyo, Japan}
    \newline
    \href{https://hennge.com/global/}{\textbf{www.hennge.com/global}}
    \begin{itemize}
        \item Introduced Prometheus for monitoring Amazon ECS clusters to alert DevOps team via Slack in case of emergency
        \textbf{[IaC, AWS, Terraform, Docker, Prometheus, Alertmanager, PromQL]}
% \item Underwent a 1-week DevOps training to implement and host a Twitter-clone on AWS using Infrastructure-as-Code
%     \textbf{[IaC, AWS, Terraform, Docker]}
    \end{itemize}

    \medskip

    \cveventflat
    {Software Engineer at }
    {\textbf{FinApps}, \textit{Almaty, Kazakhstan}}
    {Aug. 2017 -- Feb. 2018}
    {Almaty, Kazakhstan}
    \newline
    \href{https://torgai.com/}{\textbf{www.torgai.com}}
    \begin{itemize}
        \item Developed point-of-sale Android app to help local entrepreneurs digitize their businesses and achieved more than 100 customers by 2017
        \textbf{[Kotlin, Java, SQL, Firebase, Espresso, Android]}
        \item Optimized database full-text search from ~1 sec per query to less than 0.1 sec
        \textbf{[SQL, SQLite, Android]}
        \item Created face recognition system to bind customers to their checks as a part of customer loyalty program
        \textbf{[Python, Java, JAX-RS, SQL]}
    \end{itemize}

    \large
    \cvsection{Education}

    \cveventeducation
    {B.S. in Physics and Computer Science at }
    {\textbf{Ulsan National Institute of Science and Technology (UNIST)}}
    {Sept. 2014 - Aug. 2019}
    {\href{https://www.unist.ac.kr/about-unist/}{\textbf{www.unist.ac.kr/about-unist}}
        \begin{itemize}
            \item {UNIST Global Dream Scholarship: free tuition + 300\$ monthly allowance}
            \item {Cum Laude: 3.69/4.0 GPA in Computer Science courses}
        \end{itemize}}

    \cvsection{Projects}
    \cveventflat
    {\textbf{Unshaky}, }
    {{\href{https://www.github.com/DaurenAbd/Unshaky/}{www.github.com/DaurenAbd/Unshaky}}}
    {March 2018 -- June 2018}{}-
    \begin{itemize}
        \item Developed a Book reader with screen stabilization feature using accelerometer and gyroscope sensors which allows users to read while walking or in a bus
        \textbf{[Android, Kotlin]}
    \end{itemize}

    \cvsection{Competitive Programming}
    \parbox{0.45\textwidth}{
% \cvachievement
%    {}
%    {Google Foobar Challenge}
%    {2019 (5-th level)}
        \cvachievement
        {}
        {ACM-ICPC South Korea First Round}
        {2018, 2016, 2015, 2014 (participant)}
        \cvachievement
        {\faTrophy}
        {Asia-Pacific Informatics Olympiad}
        {2014 (\textbf{bronze}), 2013 (participant)}}
    \parbox{0.5\textwidth}{
% \cvachievement
%    {\faTrophy}
%    {All-Russian Team Olympiad in Informatics}
%    {2013 (\textbf{bronze}), 2011}
        \cvachievement
        {}
        {Kazakhstan IOI reserves member}
        {2012 - 2014}
% \cvachievement
%    {\faTrophy}
%    {Kazakhstan National Informatics Olympiad}
%    {2014 (\textbf{silver}), 2013 (\textbf{silver}), 2012 (\textbf{silver})}
        \cvachievement
        {\faTrophy}
        {Eurasian Informatics Olympiad}
        {2013 (\textbf{silver}), 2011 (\textbf{bronze})}}
% \bfseries\textcolor{emphasis}{}}
% \cvachievement
%     {\faTrophy}
%     {Kazakhstan National Informatics Olympiad}
%     {2014 (\textbf{silver}), 2013 (\textbf{silver}), 2012 (\textbf{silver})}}

    \clearpage

\end{document}
